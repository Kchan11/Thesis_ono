\documentclass[12pt]{jreport}
\usepackage[dvipdfmx]{graphicx}

\usepackage[bachelor]{AIcover}	% 卒論の場合

\usepackage{AIthesis}
\usepackage{geometry}

\input personal

\begin{document}
\makeCoverPageIII
\newpage

\newgeometry{margin=0mm}
\begin{figure}
  \centering
  \includegraphics[width=21.0cm]{abst.pdf}
\end{figure}

\restoregeometry

\pagestyle{plain}
% 目次
\pagenumbering{roman}	%目次のページ番号(ローマ字)
\tableofcontents 		%目次作成
\newpage
\pagenumbering{arabic} 	%本編ページ番号

% Chapter1 : はじめに
\chapter{はじめに}
ここに「はじめに」を書く.

\section{論文の書式}
論文は,A4 版の PDF データで提出する.
LaTeX で作成する場合は,このサンプル PDF を作成するのに用いた TeX ファイルを編集し作成するとよい.
LaTeXの参考書には\cite{rakuraku,bibunsyo}がある.
この PDF ファイルの構成(表紙,概要,論文本体)と見た目に準じたものであれば
Word 等を用いて作成してもよい.
論文作成にあたっては指導教員の指示を仰ぐこと.

\subsection{使用言語}
論文を記述するのに使用する言語は,日本語または英語とする.
%
\subsection{ページのレイアウト}
製本その他読みやすさ等を考慮して,マージンは大きめにとること.
以下は最低値の目安である.

\vspace*{3mm}
\hspace*{10mm}
\begin{tabular}{ll}
上マージン & 25mm 程度 \\
下マージン & 30mm 程度 (ページ番号はマージン内) \\
左マージン & 25mm 程度 \\ %(製本の都合上 30mm 以上は必要) \\
右マージン & 25mm 程度 \\
\end{tabular}

\subsection{文字の大きさ}
読みやすさ等を考慮して,極端に小さい文字や大きな文字はさけ,行間は十分にあけること. 
文字サイズ 11-12pt, 1ページ 30 行で日本語の場合は 1 行あたり 40 文字程度が目安となる. 



\chapter*{謝辞}
ここに謝辞を書く.お世話になった人,物,サービス,組織に感謝を述べる.


\renewcommand{\bibname}{参考文献}
%%% 本格的に文献管理をするなら BibTex を使いましょう

\begin{thebibliography}{99}

\bibitem{rakuraku}
野寺隆志,
\newblock 
楽々LATEX (第2版),
\newblock 
共立出版,1994.

\bibitem{bibunsyo}
奥村晴彦,黒木裕介,
\newblock 
\LaTeXe 美文書作成入門 第7版,
\newblock
技術評論社,2017.

\end{thebibliography}


\chapter*{付録}
付録があればここに書く.


\end{document}



シンプルなNTRUの計算手順\cite{MKFHEpdf1}は,以下の通りである.
\begin{enumerate}
    \item 多項式の次数$n$と法$q$を決める.$n$は2のべき乗であり,$q$は素数である.
    \item 係数の小さい多項式$f'$と$g$を生成し,$f=2f'+1$と定義する.
    \item 秘密鍵を$sk=f$,公開鍵を$pk=[2gf^{-1}]_q$と定義する.なお,下付き文字の$q$は係数を$q$で割った余りとするという意味である.
    \item 暗号化したいデータを$m\in{0,1}$の1ビットで表す.また,係数の小さい多項式$s$,$e$を生成する.暗号文は,
    \begin{equation*}
        c=[hs+2e+m]_q
    \end{equation*}
    と定義される.
    \item $\mu=[fc]_q$とし,$\mu$を2で割った剰余が復号された平文である.
\end{enumerate}
ここで,この手順をマルチキーに拡張することを考える.ユーザ1の公開鍵$h_1=[2g_1f_1^{-1}]$で暗号化した暗号文を$c_1=[h_1s_1+2e_1+m_1]_q$,ユーザ2の公開鍵$h_2=[2g_2f_2^{-1}]$で暗号化した暗号文を$c_2=[h_2s_2+2e_2+m_2]_q$とし,秘密鍵の積を
$f_{com}=f_1f_2$と定義する.すると,暗号文の和$c_{add}=[c_1+c_2]_q$を復号した結果は,
\begin{equation*}
    [f_1f_2(c_1+c_2)]_q=[2e_{add}+f_1f_2(m_1+m_2)]_q
\end{equation*}
と変形できる.ここで,右辺に対し先程のNTRUの復号手順の通り2の剰余を取ると,第一項目は$2e_{add}\equiv0(mod 2)$であり,第二項目に関しては秘密鍵$f=2f'+1$と定義していたから,当然$f_1\equiv1(mod 2)$,$f_2\equiv1(mod 2)$である.したがって,$f_1f_2\equiv1$であるから,項全体としては$m_1+m_2$のみが残る.
ゆえに,$c_{add}=[c_1+c_2]_q$が成り立つ.同様の変形を経て,積の準同型性$c_{mult}=[c_1c_2]_q$も証明できる.\\
openFHEには,NativePolyという多項式を表現できるクラスが既に実装されている.これを用いることで,これまでに示してきたMKFHEの手順をC++コードに実装した.


