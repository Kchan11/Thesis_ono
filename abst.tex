
\documentclass{AIabst} %卒論はこちら
\usepackage{amsmath}
\usepackage{latexsym}
\usepackage{enumitem}
\usepackage{hhline}

%\documentclass[master]{AIabst} %修論はこちら
% 論文概要
\input personal
\begin{document}
\makeAbstHeader
%
%
%
\section{はじめに}
昨今では様々な暗号化方式が存在するが,その中でも格子基盤暗号に分類されるNTRU方式は,計算が高速かつ量子攻撃にも耐性があるという点で優れている.また,FHEとは暗号化したデータを復号することなく直接演算を行うことができる
技術であるが,異なる鍵で暗号化した複数のデータに対して演算を行うマルチキーFHE(MKFHE)を利用できるライブラリは,Go言語で実装されているtfhe-go等,ごく限られている.
本研究では,既存のライブラリであるOpenFHEを利用し,NTRU方式に基づくMKFHEシステムを構築し,またこれを利用した全加算器を実装した.

\section{従来の暗号化方式とMKFHEの比較}
以前より利用されてきた基本的な公開鍵暗号方式は,データを暗号化したまま演算を行っても正しい平文の結果が得られる準同型性を有していることが多い.
しかしこのような従来の暗号化方式は,異なる鍵で暗号化したデータ同士の演算を想定していない.マルチキーFHEではこれを改良し,異なる鍵で暗号化したデータ同士に演算を行っても,正しい復号結果を得ることができる.\cite{MITfhe}
このような改良によって,例えば複数のユーザそれぞれが自分の鍵ペアの公開鍵でデータを暗号化し,データの内容を誰にも明かさず集計及び演算を行えるため,セキュリティを維持しつつデータを利活用することが期待できる.

%\section{ワンタイムパッドとは}
%ワンタイムパッドは,共通鍵暗号方式のひとつである.送信者は送りたいデータと同じ長さの乱数列を生成し,排他的論理和等を使って当該データと組み合わせる.ここで使用した乱数列を何らかの方法で受信者と共有しておくことで,受信者はデータを復号することができる.
%一度使用した乱数列は破棄されるため,これが流出しない限りは安全にデータを送受信することができる.しかしながらこの暗号方式には,配送したいデータと同じ長さの乱数列を用意する必要がある,すなわち秘密鍵が非常に長くなってしまう可能性があることや,
%その秘密鍵をどうやって安全に相手に受け渡すかといったような問題点も存在する.本研究では,暗号化に使用する乱数列そのものを直接受け渡すのではなく,乱数のシードを参加者全員で共有するようにした.また,シードの生成方法としては,マルチキーFHEの仕組みを応用し参加者それぞれが手元で生成した
%ランダムなビット(シードの断片)を自分の鍵で暗号化し,それらを暗号文の状態のままXOR演算を行い復号したものをシードとする.このようにして,ワンタイムパッドにおける鍵配送コストを低減するとともに安全な鍵共有を実現する.


\section{MKFHEの全加算器への応用}
異なる鍵で暗号化したデータに加算・乗算が行えるというMKFHEの仕様は,加算器に応用することが可能である.
全加算器において,考えるべき点は(1)前の桁からの繰り上がりを考慮した現在の桁の加算,(2)現在の桁の繰り上がりの計算,の2点であるが,(1)は

\begin{equation*}
    S=A\oplus B\oplus C_{in}
\end{equation*}
のように単純な3つの項の加算で表され,(2)はmod2の世界において

\begin{equation*}
    c_{carry}=(A\oplus C_{in})*(B*C_{in})\oplus C_{in}
\end{equation*}
のように,加算と乗算を利用した式に変形することができる.以上のことから,加算器の実装にはMKFHEの準同型性を利用することができる.



\section{実験手法}
まず,1ビットのデータ$a,b,c$をランダムに設定し,加算(排他的論理和),乗算(論理積),及び2つを組み合わせた式をすべて暗号文の状態で正しく計算し復号できるかを確かめる.次に,これを用いて加算器を実装する.ランダムな1~3ビットの長さのビット列を2つ生成し,正しい結果が得られたかどうかを確認する.

\section{実験結果}
単純な1ビット同士の加算・乗算を10000回ずつ行った際の標準偏差$\sigma$,実行時間,復号の成功率をまとめて表1に示す.なお,標準偏差$\sigma$は,NTRU方式でMKFHEを実装する上で必要な多項式を構成する,係数の値のばらつき具合を表す.

\begin{table}[htb]
\begin{center}
\begin{tabular}{|c|c|c|}
\hline
標準偏差$\sigma$ & 実行時間[秒] &成功率 \\ \hhline{|=|=|=|}
\hline
 & deux  &\\
\hline
0.35 & 2139.989 & 0.9986\\
\hline
0.4 & 183.442 & 0.9862\\
\hline
\end{tabular}
\end{center}
\caption{1ビット同士の加算・乗算処理の実行結果}
\end{table}

また,次に実装した1~3ビットの加算器についても同様の結果を以下の表2に示す.
\begin{table}[htb]
\begin{center}
\begin{tabular}{|c|c|c|}
\hline
標準偏差$\sigma$ & 実行時間[秒] &成功率 \\ \hhline{|=|=|=|}
\hline
 & deux  &\\
\hline
eine & zwei &\\
\hline
0.4 & 183.442 & 0.9862\\
\hline
\end{tabular}
\end{center}
\caption{1~3ビットを対象とした加算器の実行結果}
\end{table}

\section{まとめ}
本研究では,既存のライブラリを用いてNTRUベースのMKFHEによる加算器を実装し,実験的なパラメータを用いて動作させることに成功した.今後の課題としては,より実用的なパラメータ下で動作させるためのアルゴリズム改良やパラメータの調節等が挙げられる.



{\small
\baselineskip 12pt
\begin{thebibliography}{99}

  \bibitem{MITfhe}Lopez-Alt, A., Tromer, E., \& Vaikuntanathan, V. (2017). Multikey fully homomorphic encryption and applications. SIAM Journal on Computing, 46(6), 1827–1892.


\end{thebibliography}


}
\end{document}

