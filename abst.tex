\documentclass{AIabst} %卒論はこちら
%\documentclass[master]{AIabst} %修論はこちら
% 論文概要
\input personal
\begin{document}
\makeAbstHeader
%
%
%
\section{はじめに}
昨今では様々な暗号化方式が存在するが,その中でも格子基盤暗号に分類されるNTRU方式は,計算が高速かつ量子攻撃にも耐性があるという点で優れている.また,FHEとは暗号化したデータを復号することなく直接演算を行うことができる
技術であるが,異なる鍵で暗号化した複数のデータに対して演算を行うマルチキーFHE(MKFHE)を利用できるライブラリは,Go言語で実装されているtfhe-go等,ごく限られている.\\
本研究では,C++で既に実装されているopenFHEというライブラリを使用し,NTRU方式に基づくMKFHEシステムを構築し,さらにこれを応用しワンタイムパッドという暗号方式を実装する.

\section{MKFHEのアルゴリズム}
MKFHEの計算手順\cite{MKFHEpdf1}は,以下の通りである.
\begin{enumerate}
    \item 多項式の次数$n$と法$q$を決める.$n$は2のべき乗であり,$q$は素数である.
    \item 係数の小さい多項式$f'$と$g$を生成し,$f=2f'+1$と定義する.
    \item 秘密鍵を$sk=f$,公開鍵を$pk=[2gf-1]_q$と定義する.なお,下付き文字の$q$は係数を$q$で割った余りとするという意味である.
    \item 暗号化したいデータを$m\in{0,1}$の1ビットで表す.
    \item 
\end{enumerate}

論文本体作成に必要なファイルは,
\begin{center}
\begin{tabular}{ll}
{\tt personal.tex\/} & 個人データファイル\\
{\tt main.tex\/} & 本体のソースファイル\\
{\tt abst.pdf\/} & 概要のPDFファイル\\
{\tt AIcover.sty\/}  & 表紙類スタイルファイル\\
{\tt AIthesis.sty\/} & 論文本体スタイルファイル
\end{tabular}
\end{center}
である.論文本体作成時には{\tt main.tex\/}をコンパイルすればよい.

学生番号,氏名,論文タイトル,指導教員は
\begin{center}
\begin{tabular}{l}
{\tt personal.tex\/} \\
\end{tabular}
\end{center}
を編集して記入する.
概要ならびに論文本体は,
\begin{center}
\begin{tabular}{l}
{\tt abst.tex\/} \\
{\tt main.tex\/}
\end{tabular}
\end{center}
を編集すれば作成できる\footnote{論文本体をコンパイルするときに,先にコンパイルして出力された概要のPDFを読み込み使用する.もし,main.texをコンパイルして abst.pdf が見つからずエラーとなる合は,先に abst.tex をコンパイルしてダミーの概要ページを作成する.}.
%PDFを編集可能なソフトウェアを利用し,表紙と論文本体の間に概要を挿入すること.}. 
なお,表題が 2 行にまたがるときに好みの位置で改行するには,
{\tt personal.tex\/}の題目の改行位置に$\backslash\backslash$を
挿入すること.

%{\tt abst.tex\/},
%{\tt cover1.tex\/},
%{\tt cover2.tex\/},
%{\tt spine.tex\/}
%における
%名前・学生番号・指導教官・論文題目
%の細かいファイルごとの修正,特に,概要と表紙における表題の改行位置の変更は,
%それぞれのファイルの 
%ファイル読み込み部分($\backslash${\tt input\{...\}\/}の部分)
%を消去して直接書き込むことで可能である.
%この場合には,ミスが無いように十分注意すること.

\section{注意する点}
本スタイルファイルで注意する点は以下の通りである.
\begin{enumerate}
\item
基本的に通常の LaTeX と同じように利用できる.
ただし,パッケージは最低限のものしか入れていないので,
必要に応じて自身で{\tt abst.tex\/}へ追加する.
\item
見出しは{\tt section\/}と{\tt subsection\/}
しか使えない.
\item
{\tt baselineskip\/}は変更しないこと.
\item
参考文献を加える方法は,通常通りである.
例えば,
LaTeX の参考書には\cite{rakuraku,bibunsyo}がある.
\end{enumerate}

{\small
\baselineskip 12pt
\begin{thebibliography}{1}

\bibitem{rakuraku}
野寺隆志,
\newblock 
楽々LATEX (第2版),
\newblock 
共立出版,1994.

\bibitem{bibunsyo}
奥村晴彦,黒木裕介,
\newblock 
\LaTeXe 美文書作成入門 第7版,
\newblock
技術評論社,2017.

\end{thebibliography}
}
\end{document}

