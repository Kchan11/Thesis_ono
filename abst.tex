
\documentclass{AIabst} %卒論はこちら
\usepackage{amsmath}
%\documentclass[master]{AIabst} %修論はこちら
% 論文概要
\input personal
\begin{document}
\makeAbstHeader
%
%
%
\section{はじめに}
昨今では様々な暗号化方式が存在するが,その中でも格子基盤暗号に分類されるNTRU方式は,計算が高速かつ量子攻撃にも耐性があるという点で優れている.また,FHEとは暗号化したデータを復号することなく直接演算を行うことができる
技術であるが,異なる鍵で暗号化した複数のデータに対して演算を行うマルチキーFHE(MKFHE)を利用できるライブラリは,Go言語で実装されているtfhe-go等,ごく限られている.\\
本研究では,C++で既に実装されているopenFHEというライブラリを利用し,NTRU方式に基づくMKFHEシステムを構築し,さらにこれを応用しワンタイムパッドという暗号方式を改良する.

\section{従来の暗号化方式とMKFHEの比較}
以前より利用されてきた,RSAに代表されるような基本的な公開鍵暗号方式は,データを暗号化したまま加算や乗算を行っても正しい平文の結果が得られるという準同型性を有していることが多い.
しかしながらこのような従来の暗号化方式は,異なる鍵で暗号化したデータ同士の加算や乗算といった計算を想定していない.マルチキーFHEではこれを改良し,異なる鍵で暗号化したデータ同士に演算を行っても,正しい復号結果を得ることができる.
このような改良によって,例えば複数のユーザそれぞれが自分の持つ鍵ペアの公開鍵でデータを暗号化し,サーバにデータの内容を明かすことなく集計及び演算を行えるため,セキュリティを維持しつつデータを利活用することが期待できる.

\section{ワンタイムパッドとは}
ワンタイムパッドは,共通鍵暗号方式のひとつである.送信者は送りたいデータと同じ長さの乱数列を生成し,排他的論理和等を使って当該データと組み合わせる.ここで使用した乱数列を何らかの方法で受信者と共有しておくことで,受信者はデータを復号することができる.
一度使用した乱数列は破棄されるため,これが流出しない限りは安全にデータを送受信することができる.しかしながらこの暗号方式には,配送したいデータと同じ長さの乱数列を用意する必要がある,すなわち秘密鍵が非常に長くなってしまう可能性があることや,
その秘密鍵をどうやって安全に相手に受け渡すかといったような問題点も存在する.本研究では,暗号化に使用する乱数列そのものを直接受け渡すのではなく,乱数のシードを参加者全員で共有するようにした.また,シードの生成方法としては,マルチキーFHEの仕組みを応用し参加者それぞれが手元で生成した
ランダムなビット(シードの断片)を自分の鍵で暗号化し,それらを暗号文の状態のままXOR演算を行い復号したものをシードとする.このようにして,ワンタイムパッドにおける鍵配送コストを低減するとともに安全な鍵共有を実現する.



\section{注意する点}
本スタイルファイルで注意する点は以下の通りである.
\begin{enumerate}
\item
基本的に通常の LaTeX と同じように利用できる.
ただし,パッケージは最低限のものしか入れていないので,
必要に応じて自身で{\tt abst.tex\/}へ追加する.
\item
見出しは{\tt section\/}と{\tt subsection\/}
しか使えない.
\item
{\tt baselineskip\/}は変更しないこと.
\item
参考文献を加える方法は,通常通りである.
例えば,
LaTeX の参考書には\cite{rakuraku,bibunsyo}がある.
\end{enumerate}

{\small
\baselineskip 12pt
\begin{thebibliography}{1}

\bibitem{rakuraku}
野寺隆志,
\newblock 
楽々LATEX (第2版),
\newblock 
共立出版,1994.

\bibitem{bibunsyo}
奥村晴彦,黒木裕介,
\newblock 
\LaTeXe 美文書作成入門 第7版,
\newblock
技術評論社,2017.

\end{thebibliography}
}
\end{document}

